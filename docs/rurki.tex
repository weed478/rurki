\documentclass[12pt,a4paper]{mwart}

\usepackage{lmodern}
\usepackage[T1]{polski}
\usepackage[utf8]{inputenc}

\usepackage[a4paper,
            tmargin=2cm,
            bmargin=2cm,
            lmargin=2cm,
            rmargin=2cm,
            bindingoffset=0cm]{geometry}

\usepackage{tocloft}
\usepackage{hyperref}

\usepackage{amsmath}
\usepackage{amssymb}
\usepackage{siunitx}

\usepackage{graphicx}
\usepackage{subfig}
\usepackage{float}
\usepackage{booktabs}

\pdfmapfile{-mpfonts.map}

\hypersetup{
    colorlinks,
    citecolor=black,
    filecolor=black,
    linkcolor=black,
    urlcolor=black
}

\begin{document}

\title{Wibracje akustyczne warstwy materiału}
\author{Jakub Karbowski}
\date{\today}
\maketitle


\section{Sformułowanie silne}
\begin{align}
    \label{eq:main}
    -\frac{\mathrm{d}^2u}{\mathrm{d}x^2} - u &= \sin x \\
    \label{eq:war1}
    u(0) &= 0 \\
    \label{eq:war2}
    \frac{\mathrm{d}u(2)}{\mathrm{d}x} - u(2) &= 0 \\
    [0,2] \ni x &\to u(x) \in \mathbb{R} \nonumber
\end{align}
Poszukiwana funkcja to $u(x)$.

\section{Sformułowanie wariacyjne}
Równanie \eqref{eq:war1} to zerowy warunek Dirichleta
a~\eqref{eq:war2} to warunek Robina.
Za przestrzeń~$V$ 
przyjmujemy funkcje~$v(x)$ zerujące się na lewym brzegu
$v(0)=0$.
Z~\eqref{eq:war2} dostajemy~$u'(2)=u(2)$.
\begin{align*}
    -u'' - u &= \sin x \\
    -u''v - uv &= v\sin x \\
    \underbrace{-\int_0^2u''v\mathrm{d}x}_\text{przez części} - \int_0^2uv\mathrm{d}x &= \int_0^2v\sin x\mathrm{d}x \\
    -u'v|_0^2 + \int_0^2u'v'\mathrm{d}x - \int_0^2uv\mathrm{d}x &= \int_0^2v\sin x\mathrm{d}x \\
    -u'(2)v(2) + u'(0)\underbrace{v(0)}_0 + \int_0^2u'v'\mathrm{d}x - \int_0^2uv\mathrm{d}x &= \int_0^2v\sin x\mathrm{d}x \\
    -\underbrace{u'(2)}_{u(2)}v(2) + \int_0^2u'v'\mathrm{d}x - \int_0^2uv\mathrm{d}x &= \int_0^2v\sin x\mathrm{d}x \\
    \underbrace{-u(2)v(2) + \int_0^2u'v'\mathrm{d}x - \int_0^2uv\mathrm{d}x}_{B(u,v)} &= \underbrace{\int_0^2v\sin x\mathrm{d}x}_{L(v)} \\
\end{align*}

\pagebreak
\section{Program}
Program można włączyć na 2-sposoby:
\begin{enumerate}
    \item jednorazowo: \verb|julia rurki.jl|
    \item interaktywnie:
    \begin{itemize}
        \item włączyć REPL \verb|julia|
        \item \verb|julia> include("rurki.jl")|
        \item \verb|julia> rurki.main(5)|
        \item \verb|julia> rurki.main(50)|
    \end{itemize}
\end{enumerate}
Interaktywnie można wygodniej i~szybciej
testować różne parametry.
Włączanie sposobem 1.~za każdym razem
musi JIT kompilować kod. Interaktywnie
dzieje się to tylko raz.

\section{Wyniki}


\end{document}